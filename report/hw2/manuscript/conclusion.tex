\section{Discussion}


\section{Conclusion}
Dentre os resultados obtidos o não foram constatadas grandiosas diferenças dentre os resultados dos modelos, isso favorece a utilização de modelos mais simples e de menor custo computacional, em detrimento da pouca melhora da predição com o aumento do custo. Portanto ao menos que de fato haja uma necessidade em obter o melhor resultado possível, a solução com melhor custo-benefício foi a regressão linear simples, visto que é pouco custosa e obteve resultados não muito inferiores aos seus correntes, mas caso o custo computacional não precise ser considerado, o melhor modelo para o conjunto de dados fornecido é o de regressão ``ridge'', pois esse foi o que obteve os melhores resultados tanto na validação quanto no conjunto de teste.


\section{Further Work}
