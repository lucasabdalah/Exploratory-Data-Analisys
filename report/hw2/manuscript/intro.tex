\section{Introduction}

The solubility is defined by the International Union of Pure and Applied Chemistry (IUPAC) as "the analytical composition of a saturated solution, expressed in terms of the proportion of a designated solute in a designated solvent, is the solubility of that solute."~\cite{Book1}, i.e, the physical property of substances to dissolve, or not, in a given liquid. The solubility may be expressed as a concentration, molality, mole fraction, mole ratio, etc. 

This characteristic property of a specific solute-solvent combination plays a fundamental role in scientific research and practical applications, observed in medicines and drugs field, as an index to characterize chemical substances and compounds, and to assist in the identification of a compound polarity.

The popular Kosher or kitchen salt (NaCl) is formed by ionic bonds between Sodium (Na$^+$) and Chlorine (Cl$^-$) ions, resulting in a polar salt that is soluble in aaqueous solution. This is due to its structure, composition, charge density, intermolecular attraction forces, carbon chain size (in the case of organic compounds).

For an analysis where the exact influence of each chemical and physical characteristic of the experiment compounds on the solubility of the product is not known, it is convenient to study the predictors and its impact to the output at each observation. However, different approaches may impose great computational cost, in such way that a study exploiting all the data description available may keep redundant information, and lead to drawbacks in data representation and visualization. The main goal is to build framework capable of providing preprocessed data to use in a linear regression-based prediction model able to predict the data solubility. These steps precede more complex analysis such as feature extraction, dimensionality reduction, clustering and class separation~\cite{Hastie2009, Kuhn2013, James2013}.

The importance of these proposals is related to the applications, which can be in several areas, for example: with the financial market, economy in the large-scale use in the chemical industry, in the production in pharmaceutical laboratories, and even in the study of the influence of pollution on coral destruction.

Multiples approaches for the scenario are exploited in the literature with the same dataset or very similar, which linear regression models and neural networks are applied. Taking advantage on these mathematical tools, some papers explore the relationship of molecular topology predictors to infer the solubility of organic compounds in water~\cite{Artigo1}, or in a similar approach, in addition to molecular topology, electronic interactions are explored in the so-called ``E-state''~\cite{Artigo2}. For another dataset, yet working with the same chemical context and characteristics that propose to relate mineral solubility as a function of ionic strength and temperature~\cite{Artigo3}. 

In order to assess this method and overcome these limitations, we propose the application of statistics and linear algebra techniques to observe the relationship between predictors (chemical structure of compounds) and solubility. A model proposition that aims to obtain some pattern of behavior of this relationship, in order to support the prediction for unknown compounds and reduce complexity. 