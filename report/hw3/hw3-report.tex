\documentclass[conference]{IEEEtran}
\IEEEoverridecommandlockouts
% The preceding line is only needed to identify funding in the first footnote. If that is unneeded, please comment it out.
\usepackage{cite}
\usepackage{amsmath,amssymb,amsfonts}
\usepackage{algorithmic}
\usepackage{graphicx}
\usepackage{textcomp}
\usepackage{xcolor}
% User packages
\usepackage{hyperref}
% \usepackage[brazilian]{babel}

% Default arameters
\def\BibTeX{{\rm B\kern-.05em{\sc i\kern-.025em b}\kern-.08em
    T\kern-.1667em\lower.7ex\hbox{E}\kern-.125emX}}

% User parameters
\newcommand{\reviewUrgent}[1]{{\color{red} #1}} % This command is used for mandatory changes
\newcommand{\reviewNormal}[1]{{\color{yellow} #1}} % This command is used for a strong suggestion
\newcommand{\reviewMinor}[1]{{\color{green} #1}} % This command is used for minor changes suggestion

\begin{document}

\title{Conference Paper Title}

\author{\IEEEauthorblockN{Brewton Morais}
\IEEEauthorblockA{\textit{DETI} \\
\textit{Universidade Federal do Ceará}\\
Fortaleza, Brazil \\
\href{mailto:brewtonlmorais@gmail.com}{brewtonlmorais@gmail.com}}
\and
\IEEEauthorblockN{Cléber Lucas }
\IEEEauthorblockA{\textit{DETI} \\
\textit{Universidade Federal do Ceará}\\
Fortaleza, Brazil \\
\href{mailto:seuemail@email.com}{seuemail@email.com} \reviewUrgent{Update}}
\and
\IEEEauthorblockN{Lucas Abdalah}
\IEEEauthorblockA{\textit{DETI} \\
\textit{Universidade Federal do Ceará}\\
Fortaleza, Brazil \\
\href{mailto:lucasabdalah@alu.ufc.br}{lucasabdalah@alu.ufc.br}}
}

\maketitle

\begin{abstract}
An analysis of classification models was done for a set of data with informations related
to the request and approvals for research and academic activities. Using MATLAB 2022a, 
we trained different classification models, statistical and decision trees, such as 
Logistic Regression, KNN, LDA, CART, etc, making use of their respective confusion 
matrix as the results demonstrations. One of the goals these tests is to investigate if 
there are linear relationship between the provided predictors of the data base, in 
order to optimize the model that could be used by an university or a governamental 
instituion. Finally, we perform a comparison of the results and of the decision making
about which model we understand to be the most efficient to the problem.  

\end{abstract}

\begin{IEEEkeywords}
classification model, logistic regression, grand application, KNN
\end{IEEEkeywords}

\section{Introduction}



\reviewNormal{This command is used for a strong suggestion.}

\reviewMinor{This command is used for minor changes suggestion.}

Classification models are a class of mathematical models constantly used in problems of
assimilating observations of certain events to certain categories that define the problem.
Nowadays, these models are considered tools of fundamental importance in the construction 
of Deep Learning and Machine Learning algorithms. To begin, it's important to evidence 
the main existing difference between this new class of models and the class of 
regression models which is the prediction of a qualitative variable instead of a quantitative
one. This new class of tools present various practical applications, such as the 
development of a detection spam filter for emails based on the sender and on the content
of the message, the development of classification techniques of a cell belonging to tumors, 
as beningn or malignant and on the development of a model of credit release for financing.

In addition to these pure classification models applications, there are mixed applications 
techniques that combine Data Mining techniques with some other types of models to perform
a prediction. Some examples of models that use Data Mining techniques to improve their 
results are addressed by Sidropoulos \reviewNormal{Add reference}, such as Web Mining/Search 
tensor models and Brain Data Analysis. 

Meanwhile, in the work developed in this paper, some of the most used techniques in the
development of classification models were approached, such as SVM, KNN and CART, to train and 
test a funding model that will separate observations into one of two available groups:
Positive Founding and Negative Founding.


\section{Method}

\subsection{Data set}

The data used for the construction of the predictive models consists of 8708 samples
of different requests for funding from universities around the world, to finance research,
with the outcome being the success or failure of the request. The data set contains samples
from the years 2005 to 2008, with a total of 1882 predictors (independent variables). 
6633 samples from 2005 to 2007 and 1552 from 2008 were used for model training, and the 
remaining 518 samples from 2008 were used for testing the obtained model. Predictors can be 
separeted between continuous, such as the number of successes and failures passed by the 
"chief investigator", and categorical ones, such as the monetary value of the grant, 
divided into 17 groups of increasing amounts, and the month of application.

\subsection{Pre-processing}

Initially, the first step is verifying the data skewness, in case there is a strong 
tendence to the left or to the right, an adequate transformation would be applied in order to 
remove the skewness. The next step is to scale and center the data around the mean. It's 
done so since different predictors can have different scales, and if they're not normalized,
models sensitive to the variance would be affected negatively, making it biased to those
predictors with the highest values. 

Then, what we should do is to verify which predictors have actual importance to the model 
construction, that is, which of them have a stronger say for the final prediction. We can 
study this by analysing their correlation. Those with a correlation larger than 0.99, with
zero variance or sparse, that is, that have lots of zero values as data, were removed.

Then, the final approach is to verify the linearity of the predictors together with the output. 
This step is essential, and the reason for this is, once we have analysed this aspect, we can
infer if using a linear model is the adequate way of resolving this problem. For example, 
if there are too much predictos with non-linear relationship with the final output, it makes no 
sense to insist in linear prediction models.

\subsection{Cross-Validation}

Cross-validation consists in a validation technique used to validate the model with the
test set, usually taking into account the model flexibility and the mean squared error (MSE).
Shortly, it divides the data set into $k$ distinct subsets of size as equal as 
possible. From these groups, one of them is put aside to be used as validation set,
while the model is trained based on the remaining $k-1$ subsets. Once the model is trained,
the first removed subset is used as validation as previously stated. Then, the removed 
subset is restored to the principal set, and the following subset is put aside to 
perform the same procedure until all of the $k$ subgroups are all used as validation set. 
This approach improves the model capability of generalization, once it's trained with 
all the data at dispose, it also makes the error estimation more robust.

This strategy generally serves to indicate which models have a better prediction capability
on the test set, since it enables the comparison between the error levels and the variance
generated.

It's important to state that if a $k$ is chosen such as it's too small, e.g, $k=2$ 
(two subgroups) or too large ($k=$sample size), we are going to have, respectively, a 
strongly biased model because we let a lot of data outside the training step, and a potential
overfitting issue due the high model complexity, ocurring the model to have a high variance.
Thus, to mitigate both of the effects, normally $k=5,10$ is employed, since they present
an acceptable level of bias and variance.

\section{Model validation performance}

One of the most used metrics to measure the performance a classification model is the
Receiver Operating Characteristic curve (ROC) and the Area Under the Curve (AUC) \reviewNormal{correct initials?}.
 The ROC curve is traced in a graphic with the positive ratio in the y-axis and the negative
ratio in the x-axis, and each point of the curve is computed varying the classification
limiar. Decreasing this limiar makes the model classify more items as positive, increasing
the true positive and false positive. Increasing this limiar, causes the inverse
effect.

After calculating the ROC curve we can find the area under it. The area under the ROC curve
represents how well the model divides the two classes, as close the AUC value is from 1, 
better the model. However, this kind of analysis shouldn't be applied alone to validate
 a model's performance, since there is an information loss in the construction of the 
 graphic. Then, the dispersion table, which contains in its principal diagonal the number
 of true positives and true negatives, and in its secondary diagnoal the number of 
 false positives and false negatives, becomes an interesting analysis complement to the 
 ROC curve.

\subsection{Units}
\begin{itemize}
\item Use either SI (MKS) or CGS as primary units. (SI units are encouraged.) English units may be used as secondary units (in parentheses). An exception would be the use of English units as identifiers in trade, such as ``3.5-inch disk drive''.
\item Avoid combining SI and CGS units, such as current in amperes and magnetic field in oersteds. This often leads to confusion because equations do not balance dimensionally. If you must use mixed units, clearly state the units for each quantity that you use in an equation.
\item Do not mix complete spellings and abbreviations of units: ``Wb/m\textsuperscript{2}'' or ``webers per square meter'', not ``webers/m\textsuperscript{2}''. Spell out units when they appear in text: ``. . . a few henries'', not ``. . . a few H''.
\item Use a zero before decimal points: ``0.25'', not ``.25''. Use ``cm\textsuperscript{3}'', not ``cc''.)
\end{itemize}

\subsection{Equations}
Number equations consecutively. To make your 
equations more compact, you may use the solidus (~/~), the exp function, or 
appropriate exponents. Italicize Roman symbols for quantities and variables, 
but not Greek symbols. Use a long dash rather than a hyphen for a minus 
sign. Punctuate equations with commas or periods when they are part of a 
sentence, as in:
\begin{equation}
a+b=\gamma\label{eq}
\end{equation}

Be sure that the 
symbols in your equation have been defined before or immediately following 
the equation. Use ``\eqref{eq}'', not ``Eq.~\eqref{eq}'' or ``equation \eqref{eq}'', except at 
the beginning of a sentence: ``Equation \eqref{eq} is . . .''

\subsection{\LaTeX-Specific Advice}

Please use ``soft'' (e.g., \verb|\eqref{Eq}|) cross references instead
of ``hard'' references (e.g., \verb|(1)|). That will make it possible
to combine sections, add equations, or change the order of figures or
citations without having to go through the file line by line.

Please don't use the \verb|{eqnarray}| equation environment. Use
\verb|{align}| or \verb|{IEEEeqnarray}| instead. The \verb|{eqnarray}|
environment leaves unsightly spaces around relation symbols.

Please note that the \verb|{subequations}| environment in {\LaTeX}
will increment the main equation counter even when there are no
equation numbers displayed. If you forget that, you might write an
article in which the equation numbers skip from (17) to (20), causing
the copy editors to wonder if you've discovered a new method of
counting.

{\BibTeX} does not work by magic. It doesn't get the bibliographic
data from thin air but from .bib files. If you use {\BibTeX} to produce a
bibliography you must send the .bib files. 

{\LaTeX} can't read your mind. If you assign the same label to a
subsubsection and a table, you might find that Table I has been cross
referenced as Table IV-B3. 

{\LaTeX} does not have precognitive abilities. If you put a
\verb|\label| command before the command that updates the counter it's
supposed to be using, the label will pick up the last counter to be
cross referenced instead. In particular, a \verb|\label| command
should not go before the caption of a figure or a table.

Do not use \verb|\nonumber| inside the \verb|{array}| environment. It
will not stop equation numbers inside \verb|{array}| (there won't be
any anyway) and it might stop a wanted equation number in the
surrounding equation.

\subsection{Some Common Mistakes}\label{SCM}
\begin{itemize}
\item The word ``data'' is plural, not singular.
\item The subscript for the permeability of vacuum $\mu_{0}$, and other common scientific constants, is zero with subscript formatting, not a lowercase letter ``o''.
\item In American English, commas, semicolons, periods, question and exclamation marks are located within quotation marks only when a complete thought or name is cited, such as a title or full quotation. When quotation marks are used, instead of a bold or italic typeface, to highlight a word or phrase, punctuation should appear outside of the quotation marks. A parenthetical phrase or statement at the end of a sentence is punctuated outside of the closing parenthesis (like this). (A parenthetical sentence is punctuated within the parentheses.)
\item A graph within a graph is an ``inset'', not an ``insert''. The word alternatively is preferred to the word ``alternately'' (unless you really mean something that alternates).
\item Do not use the word ``essentially'' to mean ``approximately'' or ``effectively''.
\item In your paper title, if the words ``that uses'' can accurately replace the word ``using'', capitalize the ``u''; if not, keep using lower-cased.
\item Be aware of the different meanings of the homophones ``affect'' and ``effect'', ``complement'' and ``compliment'', ``discreet'' and ``discrete'', ``principal'' and ``principle''.
\item Do not confuse ``imply'' and ``infer''.
\item The prefix ``non'' is not a word; it should be joined to the word it modifies, usually without a hyphen.
\item There is no period after the ``et'' in the Latin abbreviation ``et al.''.
\item The abbreviation ``i.e.'' means ``that is'', and the abbreviation ``e.g.'' means ``for example''.
\end{itemize}
An excellent style manual for science writers is \cite{b6}.

\subsection{Authors and Affiliations}
\textbf{The class file is designed for, but not limited to, six authors.} A 
minimum of one author is required for all conference articles. Author names 
should be listed starting from left to right and then moving down to the 
next line. This is the author sequence that will be used in future citations 
and by indexing services. Names should not be listed in columns nor group by 
affiliation. Please keep your affiliations as succinct as possible (for 
example, do not differentiate among departments of the same organization).

\subsection{Identify the Headings}
Headings, or heads, are organizational devices that guide the reader through 
your paper. There are two types: component heads and text heads.

Component heads identify the different components of your paper and are not 
topically subordinate to each other. Examples include Acknowledgments and 
References and, for these, the correct style to use is ``Heading 5''. Use 
``figure caption'' for your Figure captions, and ``table head'' for your 
table title. Run-in heads, such as ``Abstract'', will require you to apply a 
style (in this case, italic) in addition to the style provided by the drop 
down menu to differentiate the head from the text.

Text heads organize the topics on a relational, hierarchical basis. For 
example, the paper title is the primary text head because all subsequent 
material relates and elaborates on this one topic. If there are two or more 
sub-topics, the next level head (uppercase Roman numerals) should be used 
and, conversely, if there are not at least two sub-topics, then no subheads 
should be introduced.

\subsection{Figures and Tables}
\paragraph{Positioning Figures and Tables} Place figures and tables at the top and 
bottom of columns. Avoid placing them in the middle of columns. Large 
figures and tables may span across both columns. Figure captions should be 
below the figures; table heads should appear above the tables. Insert 
figures and tables after they are cited in the text. Use the abbreviation 
``Fig.~\ref{fig}'', even at the beginning of a sentence.

\begin{table}[htbp]
\caption{Table Type Styles}
\begin{center}
\begin{tabular}{|c|c|c|c|}
\hline
\textbf{Table}&\multicolumn{3}{|c|}{\textbf{Table Column Head}} \\
\cline{2-4} 
\textbf{Head} & \textbf{\textit{Table column subhead}}& \textbf{\textit{Subhead}}& \textbf{\textit{Subhead}} \\
\hline
copy& More table copy$^{\mathrm{a}}$& &  \\
\hline
\multicolumn{4}{l}{$^{\mathrm{a}}$Sample of a Table footnote.}
\end{tabular}
\label{tab1}
\end{center}
\end{table}

\begin{figure}[htbp]
\centerline{\includegraphics{fig1.png}}
\caption{Example of a figure caption.}
\label{fig}
\end{figure}

Figure Labels: Use 8 point Times New Roman for Figure labels. Use words 
rather than symbols or abbreviations when writing Figure axis labels to 
avoid confusing the reader. As an example, write the quantity 
``Magnetization'', or ``Magnetization, M'', not just ``M''. If including 
units in the label, present them within parentheses. Do not label axes only 
with units. In the example, write ``Magnetization (A/m)'' or ``Magnetization 
\{A[m(1)]\}'', not just ``A/m''. Do not label axes with a ratio of 
quantities and units. For example, write ``Temperature (K)'', not 
``Temperature/K''.

\section*{Acknowledgment}

The preferred spelling of the word ``acknowledgment'' in America is without 
an ``e'' after the ``g''. Avoid the stilted expression ``one of us (R. B. 
G.) thanks $\ldots$''. Instead, try ``R. B. G. thanks$\ldots$''. Put sponsor 
acknowledgments in the unnumbered footnote on the first page.

\section*{References}

Please number citations consecutively within brackets \cite{b1}. The 
sentence punctuation follows the bracket \cite{b2}. Refer simply to the reference 
number, as in \cite{b3}---do not use ``Ref. \cite{b3}'' or ``reference \cite{b3}'' except at 
the beginning of a sentence: ``Reference \cite{b3} was the first $\ldots$''

Number footnotes separately in superscripts. Place the actual footnote at 
the bottom of the column in which it was cited. Do not put footnotes in the 
abstract or reference list. Use letters for table footnotes.

Unless there are six authors or more give all authors' names; do not use 
``et al.''. Papers that have not been published, even if they have been 
submitted for publication, should be cited as ``unpublished'' \cite{b4}. Papers 
that have been accepted for publication should be cited as ``in press'' \cite{b5}. 
Capitalize only the first word in a paper title, except for proper nouns and 
element symbols.

For papers published in translation journals, please give the English 
citation first, followed by the original foreign-language citation \cite{b6}.

\bibliographystyle{ieeetran}
\bibliography{refs}

\vspace{12pt}
\color{red}
IEEE conference templates contain guidance text for composing and formatting conference papers. Please ensure that all template text is removed from your conference paper prior to submission to the conference. Failure to remove the template text from your paper may result in your paper not being published.

\end{document}
