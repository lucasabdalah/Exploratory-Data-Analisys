\documentclass[conference]{IEEEtran}
\IEEEoverridecommandlockouts
% The preceding line is only needed to identify funding in the first footnote. If that is unneeded, please comment it out.
\usepackage{caption}
\usepackage{cite}
\usepackage{amsmath,amssymb,amsfonts}
\usepackage{algorithmic}
\usepackage{graphicx}
\usepackage{textcomp}
\usepackage{xcolor}
% User packages
\usepackage{hyperref}

% \usepackage[brazilian]{babel}

% Default arameters
\def\BibTeX{{\rm B\kern-.05em{\sc i\kern-.025em b}\kern-.08em
    T\kern-.1667em\lower.7ex\hbox{E}\kern-.125emX}}

% User parameters
\newcommand{\reviewUrgent}[1]{{\color{red} #1}} % This command is used for mandatory changes
\newcommand{\reviewNormal}[1]{{\color{yellow} #1}} % This command is used for a strong suggestion
\newcommand{\reviewMinor}[1]{{\color{green} #1}} % This command is used for minor changes suggestion

\begin{document}

\title{Conference Paper Title}

\author{\IEEEauthorblockN{Brewton Morais}
\IEEEauthorblockA{\textit{DETI} \\
\textit{Universidade Federal do Ceará}\\
Fortaleza, Brazil \\
\href{mailto:brewtonlmorais@gmail.com}{brewtonlmorais@gmail.com}}
\and
\IEEEauthorblockN{Lucas Abdalah}
\IEEEauthorblockA{\textit{DETI} \\
\textit{Universidade Federal do Ceará}\\
Fortaleza, Brazil \\
\href{mailto:lucasabdalah@alu.ufc.br}{lucasabdalah@alu.ufc.br}}
}

\maketitle

\begin{abstract}
This document is a model and instructions for \LaTeX.
This and the IEEEtran.cls file define the components of your paper [title, text, heads, etc.]. *CRITICAL: Do Not Use Symbols, Special Characters, Footnotes, 
or Math in Paper Title or Abstract.
\end{abstract}

\begin{IEEEkeywords}
breast cancer, diagnosis, machine learning, linear model, linear regression, cell biopsy
\end{IEEEkeywords}

\section{Introduction}



\reviewNormal{This command is used for a strong suggestion.}

\reviewMinor{This command is used for minor changes suggestion.}

Breast Cancer is one of the most commom cancer type in women behind only of Skin Cancer. In Brazil, according to 
the Instituto Nacional de Câncer (INCA) \reviewMinor{translate to English?}, approximately 59,700 women were diagnosed
with Breast Cancer in 2019, with a mortality rate of 13.68 per 100,000 habitants. 
Despite the fact that there are a limited amount of data worlwide, in some regions such as Western Europe and North America,
breast cancer is the one with the highest incidence.

Although the treatment for it is normally agressive, early-stage cancer detection may reduce the death rate in the long term \cite{b1}. 
Some exams such as mammography and contrast-enhaced (CE) digital mammography are commonly used to do the diagnosis, 
as well as the breast biopses aims to infer if the tumor is malignant or benign, requiring trained people. 

With the purpose of improving Breast Cancer Diagnosis, the analysis of potential cancerous cells have been made,
retrieving features such as texture, smoothness, radius size, etc. The study of these parameters can be crucial 
to the early-stage breast cancer detection, as it will be investigated in this work. 

Throughtout this paper, the relevance of such cell caracteristics will be analyzed by statitics metrics and data visualization tools 
with the goal of fiding a correlation or a probability relationship to the final diagnosis. The final goal is to 
build a probabilistic model capable of predicting the malignancy of tumor cells.

Therefore this work focus on the search for linear relationship between the parameters 
and the class of data, which it's refered as M for malignant and B for benign cells, wich can be 
categorized as a binary classification problem. At first, it may be possible to use 
Linear Regression techniques and Maximum Likelihood Estimator to achieve this goal, which will be 
confirmed throughout the development of the study. 

\section{Breast Cancer Dataset}

\subsection{Context and parameters}

The breast cancer is characterized when the cells in the women's breast start
growing uncontrollaby, which forms tumors that can be either bening or malign. 

The dataset contains cell parameters such as radius mean, texture mean, area mean, 
smoothness and so on, which are normally the parameters altered by a malignant
tumor. 
It is composed by 32 columns and 539 rows, that is, 32 features 
for 539 samples. There are 2 columns between the 32 that do not represent any information 
concerning the cells, but the identification of the pacients and their final
diagnosis as well. Therefore, the dataset contains two classes, Malignant (M) and Benign (B),
characterizing a binary classification problem. 

Although there are only two classes, the key challenge is to build a prediction
model based on the weight of each feature on the final result, that is, the relevance
on the final diagnosis. It's proposed a data investigation on the possible missing values as 
well as a dimensionality reduction since there are too many columns, what would 
make the model too complex and probably not able to generalize if all of these 
features were to be considered for the model construction. 

Throughout this work, the programming language \textit{\textbf{Python}} was employed 
using mainly the \textit{Pandas} library, which deals with data frame, including the 
preprocessing and cleaning steps, as well as the plots and inferences. 

\section{Data Preprocessing}
Before dealing with plots and inferences from the data, it is extremely important to 
perform a normalization in order to facilitate the visualization of the histograms 
and correlation plots. It is done by applying the z-score on the dataset.

\subsection{Z-Score Normalization}

Also called a standard score, the z-score is a normalization technique that gives the 
idea of how far from the mean a data point is. It can be placed on a 
\textit{normal distribution} curve. In order to do so, it is necessary to know the 
mean $\mu$ and the standard deviation $\sigma$ of the points.

Let $\Bar{x}$ be the sample mean,
$$\Bar{x} = \sum_i \frac{x_i}{N}$$
then
$$z_i = \cfrac{x_i - \Bar{x}}{\sigma}$$

Therefore, since now the data is located around a zero mean normal distribution with 
unitary variance, the identification of outliers has became easier.

The need to perform such normalization is that a model constructed from a 
non-normalized data set will probably have a biased result, since the model can be 
sensitive to variance. Therefore, if there are large differences between the range of 
some variables, those with the highest range will have more relevance to the model, 
since the variance is larger, that is, it carries more information.

\subsection{Data review}\label{AA}
It was check the existence of any missing values, but it was found that all data is fully completed, which eliminates the need of any data compesation technique.

\subsection{Statistical Analysis}
\begin{itemize}
\item To get a general perspective of the dataset, using the method \textit{describe} from the Pandas library, the following table is a cutout of the entire table involving all parameters, here not shown in order to not difficult the comprehension: 

\begin{table}[htbp]
\caption{Data Statistics 3 predictors}
\begin{center}
\begin{tabular}{|c|c|c|c|}
\hline

\cline{2-4} 
\textbf{Stats} & \textbf{\textit{radius mean}}& \textbf{\textit{texture mean}}& \textbf{\textit{perimeter mean}} \\
\hline
\textbf{\textit{count}}& 569 & 569 & 569  \\
\hline
\textbf{\textit{mean}}&  -1.256562 & 1.049736 & 1.272171 \\
\hline
\textbf{\textit{std}}& 1.000880 & 1.000880 & 1.000880 \\ 
\hline
\textbf{\textit{min}} &-2.029648  & -2.229249 & -1.984504 \\ 
\hline
\textbf{\textit{25\%}}& -6.893853 &-7.259631	& -6.919555 \\
\hline
\textbf{\textit{50\%}}& -2.150816 &-1.046362 & -2.359800 \\ 
\hline
\textbf{\textit{75\%}}& 4.693926	& 5.841756 &4.996769 \\ 
\hline
\textbf{\textit{max}}& 3.971288 & 4.651889 & 3.976130 \\ 
\hline
\end{tabular}
\label{tab1}
\end{center}
\end{table}

\item The following table shows the mean statistics of 3 predictors grouped by Diagnosis: malignant or benign.


\begin{table}[htbp]
\caption{Data Statistics grouped by diagnosis}
\begin{center}
\begin{tabular}{|c|c|c|c|}
\hline

\cline{2-4} 
\textbf{Diagnosis} & \textbf{\textit{radius mean}}& \textbf{\textit{texture mean}}& \textbf{\textit{perimeter mean}} \\
\hline
\textbf{\textit{B}}& 1.097064 &-2.073335 &	1.269934  \\
\hline
\textbf{\textit{M}}& 1.829821 & -0.353632 & 1.685955 \\
\hline
\end{tabular}
\label{tab1}
\end{center}
\end{table}

\end{itemize}


\section{Methods}
\reviewUrgent{Describe our methods and develloped work.}


\section{Experimental Results}
\reviewUrgent{Describe our statistic results and figures.}


\section{Discussion}
\reviewUrgent{Further description about our results and what it implies.}


\section{Conclusion}
\reviewUrgent{Quick recap about what we did, reinforce our results strengths and weakness.}

\section{Further Work}
\reviewUrgent{Quick recap about our work weakness and propose new approach to overcome its weakness.}


% \subsection{Equations}
% Number equations consecutively. To make your 
% equations more compact, you may use the solidus (~/~), the exp function, or 
% appropriate exponents. Italicize Roman symbols for quantities and variables, 
% but not Greek symbols. Use a long dash rather than a hyphen for a minus 
% sign. Punctuate equations with commas or periods when they are part of a 
% sentence, as in:
% \begin{equation}
% a+b=\gamma\label{eq}
% \end{equation}

% Be sure that the 
% symbols in your equation have been defined before or immediately following 
% the equation. Use ``\eqref{eq}'', not ``Eq.~\eqref{eq}'' or ``equation \eqref{eq}'', except at 
% the beginning of a sentence: ``Equation \eqref{eq} is . . .''

% \subsection{\LaTeX-Specific Advice}

% Please use ``soft'' (e.g., \verb|\eqref{Eq}|) cross references instead
% of ``hard'' references (e.g., \verb|(1)|). That will make it possible
% to combine sections, add equations, or change the order of figures or
% citations without having to go through the file line by line.

% Please don't use the \verb|{eqnarray}| equation environment. Use
% \verb|{align}| or \verb|{IEEEeqnarray}| instead. The \verb|{eqnarray}|
% environment leaves unsightly spaces around relation symbols.

% Please note that the \verb|{subequations}| environment in {\LaTeX}
% will increment the main equation counter even when there are no
% equation numbers displayed. If you forget that, you might write an
% article in which the equation numbers skip from (17) to (20), causing
% the copy editors to wonder if you've discovered a new method of
% counting.

% {\BibTeX} does not work by magic. It doesn't get the bibliographic
% data from thin air but from .bib files. If you use {\BibTeX} to produce a
% bibliography you must send the .bib files. 

% {\LaTeX} can't read your mind. If you assign the same label to a
% subsubsection and a table, you might find that Table I has been cross
% referenced as Table IV-B3. 

% {\LaTeX} does not have precognitive abilities. If you put a
% \verb|\label| command before the command that updates the counter it's
% supposed to be using, the label will pick up the last counter to be
% cross referenced instead. In particular, a \verb|\label| command
% should not go before the caption of a figure or a table.

% Do not use \verb|\nonumber| inside the \verb|{array}| environment. It
% will not stop equation numbers inside \verb|{array}| (there won't be
% any anyway) and it might stop a wanted equation number in the
% surrounding equation.

% \subsection{Some Common Mistakes}\label{SCM}
% \begin{itemize}
% \item The word ``data'' is plural, not singular.
% \item The subscript for the permeability of vacuum $\mu_{0}$, and other common scientific constants, is zero with subscript formatting, not a lowercase letter ``o''.
% \item In American English, commas, semicolons, periods, question and exclamation marks are located within quotation marks only when a complete thought or name is cited, such as a title or full quotation. When quotation marks are used, instead of a bold or italic typeface, to highlight a word or phrase, punctuation should appear outside of the quotation marks. A parenthetical phrase or statement at the end of a sentence is punctuated outside of the closing parenthesis (like this). (A parenthetical sentence is punctuated within the parentheses.)
% \item A graph within a graph is an ``inset'', not an ``insert''. The word alternatively is preferred to the word ``alternately'' (unless you really mean something that alternates).
% \item Do not use the word ``essentially'' to mean ``approximately'' or ``effectively''.
% \item In your paper title, if the words ``that uses'' can accurately replace the word ``using'', capitalize the ``u''; if not, keep using lower-cased.
% \item Be aware of the different meanings of the homophones ``affect'' and ``effect'', ``complement'' and ``compliment'', ``discreet'' and ``discrete'', ``principal'' and ``principle''.
% \item Do not confuse ``imply'' and ``infer''.
% \item The prefix ``non'' is not a word; it should be joined to the word it modifies, usually without a hyphen.
% \item There is no period after the ``et'' in the Latin abbreviation ``et al.''.
% \item The abbreviation ``i.e.'' means ``that is'', and the abbreviation ``e.g.'' means ``for example''.
% \end{itemize}
% An excellent style manual for science writers is \cite{b6}.

% \subsection{Authors and Affiliations}
% \textbf{The class file is designed for, but not limited to, six authors.} A 
% minimum of one author is required for all conference articles. Author names 
% should be listed starting from left to right and then moving down to the 
% next line. This is the author sequence that will be used in future citations 
% and by indexing services. Names should not be listed in columns nor group by 
% affiliation. Please keep your affiliations as succinct as possible (for 
% example, do not differentiate among departments of the same organization).

% \subsection{Identify the Headings}
% Headings, or heads, are organizational devices that guide the reader through 
% your paper. There are two types: component heads and text heads.

% Component heads identify the different components of your paper and are not 
% topically subordinate to each other. Examples include Acknowledgments and 
% References and, for these, the correct style to use is ``Heading 5''. Use 
% ``figure caption'' for your Figure captions, and ``table head'' for your 
% table title. Run-in heads, such as ``Abstract'', will require you to apply a 
% style (in this case, italic) in addition to the style provided by the drop 
% down menu to differentiate the head from the text.

% Text heads organize the topics on a relational, hierarchical basis. For 
% example, the paper title is the primary text head because all subsequent 
% material relates and elaborates on this one topic. If there are two or more 
% sub-topics, the next level head (uppercase Roman numerals) should be used 
% and, conversely, if there are not at least two sub-topics, then no subheads 
% should be introduced.

% \subsection{Figures and Tables}
% \paragraph{Positioning Figures and Tables} Place figures and tables at the top and 
% bottom of columns. Avoid placing them in the middle of columns. Large 
% figures and tables may span across both columns. Figure captions should be 
% below the figures; table heads should appear above the tables. Insert 
% figures and tables after they are cited in the text. Use the abbreviation 
% ``Fig.~\ref{fig}'', even at the beginning of a sentence.

% \begin{table}[htbp]
% \caption{Table Type Styles}
% \begin{center}
% \begin{tabular}{|c|c|c|c|}
% \hline
% \textbf{Table}&\multicolumn{3}{|c|}{\textbf{Table Column Head}} \\
% \cline{2-4} 
% \textbf{Head} & \textbf{\textit{Table column subhead}}& \textbf{\textit{Subhead}}& \textbf{\textit{Subhead}} \\
% \hline
% copy& More table copy$^{\mathrm{a}}$& &  \\
% \hline
% \multicolumn{4}{l}{$^{\mathrm{a}}$Sample of a Table footnote.}
% \end{tabular}
% \label{tab1}
% \end{center}
% \end{table}

% \begin{figure}[htbp]
% \centerline{\includegraphics{fig1.png}}
% \caption{Example of a figure caption.}
% \label{fig}
% \end{figure}

% Figure Labels: Use 8 point Times New Roman for Figure labels. Use words 
% rather than symbols or abbreviations when writing Figure axis labels to 
% avoid confusing the reader. As an example, write the quantity 
% ``Magnetization'', or ``Magnetization, M'', not just ``M''. If including 
% units in the label, present them within parentheses. Do not label axes only 
% with units. In the example, write ``Magnetization (A/m)'' or ``Magnetization 
% \{A[m(1)]\}'', not just ``A/m''. Do not label axes with a ratio of 
% quantities and units. For example, write ``Temperature (K)'', not 
% ``Temperature/K''.

% \section*{Acknowledgment}

% The preferred spelling of the word ``acknowledgment'' in America is without 
% an ``e'' after the ``g''. Avoid the stilted expression ``one of us (R. B. 
% G.) thanks $\ldots$''. Instead, try ``R. B. G. thanks$\ldots$''. Put sponsor 
% acknowledgments in the unnumbered footnote on the first page.

% \section*{References}

% Please number citations consecutively within brackets \cite{b1}. The 
% sentence punctuation follows the bracket \cite{b2}. Refer simply to the reference 
% number, as in \cite{b3}---do not use ``Ref. \cite{b3}'' or ``reference \cite{b3}'' except at 
% the beginning of a sentence: ``Reference \cite{b3} was the first $\ldots$''

% Number footnotes separately in superscripts. Place the actual footnote at 
% the bottom of the column in which it was cited. Do not put footnotes in the 
% abstract or reference list. Use letters for table footnotes.

% Unless there are six authors or more give all authors' names; do not use 
% ``et al.''. Papers that have not been published, even if they have been 
% submitted for publication, should be cited as ``unpublished'' \cite{b4}. Papers 
% that have been accepted for publication should be cited as ``in press'' \cite{b5}. 
% Capitalize only the first word in a paper title, except for proper nouns and 
% element symbols.

% For papers published in translation journals, please give the English 
% citation first, followed by the original foreign-language citation \cite{b6}.

\bibliographystyle{ieeetran}
\bibliography{refs}

% \vspace{12pt}
% \color{red}
% IEEE conference templates contain guidance text for composing and formatting conference papers. Please ensure that all template text is removed from your conference paper prior to submission to the conference. Failure to remove the template text from your paper may result in your paper not being published.

\end{document}
